\addcontentsline{toc}{section}{Introduction}
\section*{Introduction}

As second-year student in Polytech’Clermont-Ferrand engineering school, we have
to perform a placement abroad in order to discover the business world and to
possibly act up to what we have learned during the years of school. 

I made my intership in the university of Molde in Norway : "Høgskolen i Molde".
This one is specialized in logistics. During my practice in Molde, I worked
with Guillaume Lanquepin-Chesnais on lot-sizing problem. This is a
classic issue of logistics which consists to satisfy the demand and minimize
the producing cost. We need to balance setup cost and holding stock.

As this placement is research oriented, it consisted in reading and
understanding articles about the topic and implementing what I have read for a
new problem. An other purpose of this practice is to write an article which
explains what we do and our results about theses problems. That is why this
report look like a scientific article, I decided so to use research formalism
to write this report.

Hence, we can find in this report informations about the university, the goals
of this training period and the different steps of my placement with all my
studies and results.
\newpage
\section{Høgskolen i Molde} 

\subsection{The university}

Molde University College - Specialized University in Logistics is located in the city of Molde, on the Western coast of Norway.  Molde University College  was established in 1994, after merging the Regional College of Molde and the Nursing School of Molde.

\begin{figure}[H]
\includegraphics[scale=2]{image/university}
\centering
\caption{Høgskolen i Molde}
\end{figure}

The university of Molde is 2000 students, 180 employees and 32 study programs in 2 faculties:


\begin{align*}
    &\text{- Faculty of Economics, Informatics and Social Sciences }\\
    &\text{- Faculty of Health and Social Care}
\end{align*}


\subsection{Purpose of the placement}
Lot sizing is a classic issue of logistics which consists to satisfy the demand 
and minimize the producing cost. We need to balance setup cost and holding stock.

Even if there are lot of heuristics, exact methods exists. In the 50's, Wagner
and Whitin\cite{Wagner&Whitin58}, show that no producing planning can be
optimal if we product and hold in the same time it is the theorem horizon since
we are always able to cut costs by rescheduling production to avoid inventory.
There are an algorithm thus, based on the dynamic programming, which solves a
lot sizing problem in quadratic complexity of the number of period\footnote{for
each period it computes all previous production costs.} Then it was showed that
the aggregate demand and costs make a convex space where its envelop is the
optimal solution\cite{Wagelmans&Kolen92}.  This allows to search the lower cost
on this planning horizon, and this search have a $logT$ complexity that's why
we have an algorithm with a better complexity, even linear when all parameters
are constant and the search occurred in constant
complexity.

On an other side, Thomas\cite{Thomas70} extends the minimization cost of Wagner
and Whitin \cite{Wagner&Whitin58} to the maximization of profit. In fact in the
monopolistic environment the demand depend only of the price, and pricing can
be then a decision variable. For the same reason than Wagner and Within's one,
this algorithm have a quadratic complexity
 
The first goald is to use the algorithm based on the dynamic programming 
of \cite{Wagelmans&Kolen92} and to apply it in the monopolistic environment to 
adapt the algorithm of Thomas to have a best complexity.

The second goald of this placement is to add a marketing effect in the model
and to study this model. 

