\addcontentsline{toc}{section}{Introduction}
\section*{Introduction}

As a second-year student in Polytech’Clermont-Ferrand engineering school, we have to perform a placement abroad to discover professional world and to possibly act up to what we have learned during the years of school. 

I made my intership in the university of Molde in Norway : "Høgskolen i Molde". This university is specialized in logistics. During my work in Molde, I worked with Guillaume Lanquepin-Chesnais on lots sizing problem. Lot sizing is a classic issue of logistics which consists to satisfy the demand 
and minimize the producing cost. We need to balance setup's cost and hold's stock.

As this placement is more "research-type", it consisted in read and understand articles about the topic and to adapt what I have read for a new problem.
An other purpose of this placement is to write an article which explain what we do and our results about theses problems. That's why this repport look like an article, I decided to use research standart for the write of this repport.

Hence, in this repport we can find information about the university, the goals of this placement and the different step of my placement whith all my study and results.
\newpage
\section{Høgskolen i Molde} 

\subsection{The university}

Molde University College - Specialized University in Logistics is located in the city of Molde, on the Western coast of Norway.  Molde University College  was established in 1994, after merging the Regional College of Molde and the Nursing School of Molde.

\begin{figure}[H]
\includegraphics[scale=2]{image/university}
\centering
\caption{Høgskolen i Molde}
\end{figure}

The university of Molde is 2000 students, 180 employees and 32 study programs in 2 faculties:


\begin{align*}
    &\text{- Faculty of Economics, Informatics and Social Sciences }\\
    &\text{- Faculty of Health and Social Care}
\end{align*}


\subsection{Purpose of the placement}
Lot sizing is a classic issue of logistics which consists to satisfy the demand 
and minimize the producing cost. We need to balance setup's cost and hold's stock.

Even if there are a lot of heuristics, exact's methods exists. In the 50's, 
Wagner and Whitin\cite{Wagner&Whitin58}, show that no producing planning can be optimal 
if we product and hold in the same time it is the theorem horizon. There are an 
algorithm, based on the dynamic programming, which solve a lot sizing problem in 
quadratic complexity (for each period it compute all producing costs.)
Then it's showed that the aggregate demand and costs make a 
convex space where its envelop is the optimal solution\cite{Wagelmans&Kolen92}.
This allow to search the lower cost on this planning horizon, and this search 
have a $logT$ complexity that's why we have an algorithm with a best complexity, 
even linear when all parameters are constant and the search occurred in constant
complexity.

On an other side Thomas\cite{Thomas70} extends the minimization cost of Wagner
 and Whitin \cite{Wagner&Whitin58} by the maximization of profit. In fact in the
 monopolistic environment the demand depend only of the price that allow to it 
 increase benefits. But this algorithm have a quadratic complexity
 
The first goald is to use the algorithm based on the dynamic programming 
of \cite{Wagelmans&Kolen92} and to apply it in the monopolistic environment to 
adapt the algorithm of Thomas to have a best complexity.

The second goald of this placement is to add a marketing effect in the model
and to study this model. 

