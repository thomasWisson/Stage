\documentclass{beamer}
\usepackage[utf8]{inputenc}
\usepackage[frenchb]{babel}
\usepackage{float}
\usepackage{graphicx}
\usetheme{CambridgeUS}
\title[Soutenance de projet]{Soutenance de projet\\Etude des profits de production suivant divers demandes}
\author[WISSON -- GMM2]{Wisson Thomas -- GMM2}
\date{Jeudi 1er Septembre 2011}

\pgfdeclareimage[height=0.85cm]{left-logo}{image/logoPolytech}
\pgfdeclareimage[height=0.85cm]{right-logo}{image/logoHogskolen}
\logo{\pgfuseimage{right-logo}}

\setbeamertemplate{sidebar left}
{
\logo{\pgfuseimage{left-logo}}
\vfill%
\rlap{\hskip0.1cm\insertlogo}%
\vskip15pt%
}
\begin{document}

\begin{frame}
\titlepage
\end{frame}



\AtBeginSection
{
\begin{frame}<beamer>
   \frametitle{Plan}
   \tableofcontents[currentsection,currentsubsection]
\end{frame}

}


\section{Introduction}
\begin{frame}{Accueil}
    \begin{figure}[H]
        \text{Molde, Norvège}\\
        \includegraphics[scale=0.2]{image/carte}
        \includegraphics[scale=0.2]{image/map1}\\
    \end{figure}
\end{frame}

\begin{frame}{Accueil}
    \begin{itemize}
    \item Université de Molde 
    \begin{figure}[H]
        \includegraphics[scale=0.3]{image/university}\\
        \caption{Høgskolen i Molde}
    \end{figure}
        \item Tuteur : Guillaume Lanquepin-Chesnais
        \item Date du Stage : du 2 mai au 1er juillet 2011
    \end{itemize}
\end{frame}

\begin{frame}{Sujet}
    \begin{itemize}
        \item Problème de lot
        \item Ajout du marketing
    \end{itemize}
\end{frame}

\section{Problème de lots}
\begin{frame}{Introduction aux problèmes de lots}
\end{frame}

\begin{frame}{Algorithme de Wagner et Within}
\end{frame}

\begin{frame}{Algorithme de Walgner et Kolen}
\end{frame}

\begin{frame}{Algorithme de Thomas}
\end{frame}

\section{Mon travail}
\begin{frame}{Amélioration d'une complexité}
\end{frame}

\begin{frame}{Ajout de l'effet de marketing}
\end{frame}

\section{Conclusion}
\begin{frame}{Conclusion}
\end{frame}
\end{document}
